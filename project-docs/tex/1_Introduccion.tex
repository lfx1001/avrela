\capitulo{1}{Introducción}

En la actualidad, la enseñanza a distancia goza de gran importancia debido a ventajas inherentes como la deslocalización de profesores y alumnos, la libertad de agenda de los alumnos a la hora de seguir la formación y el uso eficiente de los recursos educativos que permite a los centros de formación ampliar su capacidad de alumnado al no estar sujetos a limitaciones propias de la enseñanza presencial como aulas y laboratorios.

Sin embargo, la enseñanza a distancia de materias relacionadas con la ciencia, la tecnología y  la ingeniería  presenta dificultades derivadas de la propia naturaleza de estas disciplinas: a menudo requieren prácticas en el laboratorio para dotar al alumno de las competencias necesarías . \cite{bourne2019} menciona  cómo, históricamente, algunas de las necesidades especiales de los estudios universitarios de ingeniería no han sido bien cubiertas por los métodos de enseñanza online.

La emergencia de las tecnologías de la información ha permitido el desarrollo de laboratorios virtuales. Su acceso online y el software que los soporta permite a los alumnos una reproducción cada vez más fiel del trabajo realizado en el laboratorio.

Como \cite{tejado2018}  señala, los laboratorios virtuales son esenciales para que los alumnos adquieran conocimientos prácticos que se convertirán en  activos fundamentales en su  carrera profesional. En el desarrollo de laboratorios virtuales, uno de los focos de acción es facilitar la autoevaluación para mejorar la experiencia de aprendizaje de los alumnos y reducir la carga de trabajo de los docentes.Una autoevaluación clara y concisa lleva al alumno a una reflexión sobre sus propios errores, convirtiendo una evaluación formativa en una evaluación formadora \cite{sánchez2009}.

La enseñanza de un modelo agile de gestión de proyectos mediante Github y el sistema de control de versiones Git proporciona el marco de este trabajo de fin de grado. Corresponde a la asignatura Gestión de proyectos - bloque gestión ágil, del tercer curso del grado en ingeniería informática. En esta asignatura se plantean dos actividades a modo de laboratorio virtual:

\begin{itemize}
	\item Gestión de tareas de un proyecto software. El alumno realiza una simulación de planificación de tareas de un proyecto software.
	\item Simulación de control de versiones de un proyecto open source. El alumno simula la gestión de versiones de código utilizando un proyecto disponible en un repositorio público.
\end{itemize}

Los supuestos de ambas actividadades proporcionan el  detalle paso por paso de la simulación a realizar además de la referencia al repositorio original. 

\imagen{planning-simu-showcase.png}{Gestión de tareas de un proyecto software: extracto de información para la simulación}{1}

El producto entegable de cada una de las prácticas debe constar de diversas capturas de pantalla que constaten la fidelidad de la simulación con respecto al original.Para la evaluación de cada una de las prácticas, existe una rúbrica de evaluación que actualmente es aplicada de forma manual en base al producto entregado.

\imagen{planning-simu-rubric.png}{Gestión de tareas de un proyecto software: extracto de rúbrica para la evaluación}{1}

La evaluación de ambas actividades puede automatizarse en gran medida. La información de las tareas de un proyecto y del control de versiones de Github es accesible vía API. A partir del contraste de la información del proyecto referencia y el proyecto de la simulación podemos dar contestación a gran parte de los elementos de la rúbrica. Del mismo modo, podemos explotar esta comparación para proporcionar retroalimentación al alumno de aquellos elementos de la simulación que concuerdan o discrepan con el original.

