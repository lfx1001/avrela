\capitulo{1}{Introducción}

En la actualidad, la enseñanza a distancia goza de gran importancia debido a ventajas inherentes como la deslocalización de profesores y alumnos, la libertad de agenda de los alumnos a la hora de seguir la formación y el uso eficiente de los recursos educativos que permite a los centros de formación ampliar su capacidad de alumnado al no estar sujetos a limitaciones propias de la enseñanza presencial como aulas y laboratorios.

Sin embargo, la enseñanza a distancia de materias relacionadas con la ciencia, la tecnología y  la ingeniería  presenta dificultades derivadas de la propia naturaleza de estas disciplinas: a menudo requieren prácticas en el laboratorio para dotar al alumno de las competencias necesarías .

La emergencia de las tecnologías de la información ha permitido el desarrollo de laboratorios virtuales. Su acceso online y el software que los soporta permite a los alumnos una reproducción cada vez más fiel del trabajo realizado en el laboratorio.

La enseñanza de un modelo agile de gestión de proyectos mediante Github y el sistema de control de versiones Git proporciona el marco de este trabajo de fin de grado.



 