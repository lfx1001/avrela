\capitulo{1}{Introducción}

En la actualidad, la enseñanza a distancia goza de gran importancia debido a ventajas inherentes como la deslocalización de profesores y alumnos, la libertad de agenda de los alumnos a la hora de seguir la formación y el uso eficiente de los recursos educativos que permite a los centros de formación ampliar su capacidad de alumnado al no estar sujetos a limitaciones propias de la enseñanza presencial como aulas y laboratorios.

Sin embargo, la enseñanza a distancia de materias relacionadas con la ciencia, la tecnología y  la ingeniería  presenta dificultades derivadas de la propia naturaleza de estas disciplinas: a menudo requieren prácticas en el laboratorio para dotar al alumno de las competencias necesarías .

La emergencia de las tecnologías de la información ha permitido el desarrollo de laboratorios virtuales. Su acceso online y el software que los soporta permite a los alumnos una reproducción cada vez más fiel del trabajo realizado en el laboratorio.

\cite{tejado2018}  señala que los laboratorios virtuales son son esenciales para que los alumnos adquieran conocimientos prácticos que se convertirán en  activos fundamentales para la  carrera profesional. En el desarrollo de laboratorios virtuales, uno de los focos es facilitar la autoevaluación para reducir la carga de trabajo de los docentes.

La enseñanza de un modelo agile de gestión de proyectos mediante Github y el sistema de control de versiones Git proporciona el marco de este trabajo de fin de grado. Corresponde a la asignatura Gestión de proyectos - bloque gestión ágil, del tercer curso del grado en ingeniería en informática.

Se plantea la actividad "Gestión de tareas de un proyecto software", en la que los alumnos, a partir de un proyecto existente en github y una ventana temporal que engloba varios sprints, simulan en un repositorio propio lo ocurrido.

Esta simulación comprende la creación y edición de:

\begin{itemize}
	\item Historias de usuario.
	\item Sprints.
	\item Issues.
\end{itemize}



 