\capitulo{1}{Introducción}

En la actualidad, la enseñanza a distancia goza de gran importancia debido a ventajas inherentes como la deslocalización de profesores y estudiantes y la  libertad de agenda de los estudiantes a la hora de seguir la formación.

Los centros de formación pueden  hacer un  uso más eficiente de los recursos educativos, lo que permite a ampliar su capacidad de estudiantes al no estar sujetos a limitaciones propias de la enseñanza presencial como aulas y laboratorios.

Sin embargo, la enseñanza a distancia de materias relacionadas con la ciencia, la tecnología y  la ingeniería  presenta dificultades derivadas de la propia naturaleza de estas disciplinas: a menudo requieren prácticas en el laboratorio para dotar al alumno de las competencias necesarías. \cite{bourne2019} menciona  cómo, históricamente, algunas de las necesidades especiales de los estudios universitarios de ingeniería no han sido bien cubiertas por los métodos de enseñanza online.

La emergencia de las tecnologías de la información ha permitido el desarrollo de laboratorios virtuales. Su acceso online y el software que los soporta permite a los estudiantes una reproducción cada vez más fiel del trabajo realizado en el laboratorio.

Como \cite{tejado2018}  señala, los laboratorios virtuales son esenciales para que los estudiantes adquieran conocimientos prácticos que se convertirán en  activos fundamentales en su  carrera profesional. En el desarrollo de laboratorios virtuales, uno de los focos de acción es facilitar la autoevaluación para mejorar la experiencia de aprendizaje de los alumnos y reducir la carga de trabajo de los docentes.Una autoevaluación clara y concisa lleva al alumno a una reflexión sobre sus propios errores, convirtiendo una evaluación formativa en una evaluación formadora \cite{sánchez2009}.

La enseñanza de un modelo ágil de gestión de proyectos mediante GitHub, ZenHub  y el sistema de control de versiones Git proporciona el marco de este trabajo de fin de grado. En concreto nos centraremos en la asignatura de Gestión de Proyectos en el bloque correspondiente a la Gestión ágil (ver guía docente https://ubuvirtual.ubu.es/mod/guiadocente/get_guiadocente.php?asignatura=6366_online&cursoacademico=2022):

\begin{itemize}
	\item Simulación de gestión de tareas de un proyecto software. El estudiante realiza una simulación de planificación de tareas de un proyecto software.
	\item Simulación de control de versiones de un proyecto open source. El estudiante simula la gestión de versiones de código utilizando un proyecto disponible en un repositorio público.
\end{itemize}

Los casos de estudio de simulación de ambas actividadades proporcionan el  detalle paso por paso de la simulación a realizar, como el que se muestra en la Figura~\ref{fig:planning-simu-showcase.png} y, además, el alumno cuenta con el repositorio original a modo de referencia. 

\imagen{planning-simu-showcase.png}{Descripción cuantitativa y cualitativa de la simulación de un sprint/iteración}{1}

El producto entregable es la clonación del repositorio o url del repositorio con el caso de estudio de simulación.Para la evaluación de cada uno de los casos de estudio de simulación, existe una rúbrica de evaluación que se muestra en la  Figura~\ref{fig:planning-simu-rubric.png} que actualmente es aplicada de forma manual en base al producto entregado.

\imagen{planning-simu-rubric.png}{Gestión de tareas de un proyecto software: extracto de rúbrica para la evaluación}{1}

La evaluación de ambos casos de estudio de simulación puede automatizarse en gran medida. La información de las planificación de un proyecto en GitHub y en ZenHub, así como la información  del control de versiones de GitHub, es accesible vía API. A partir del contraste de la información del proyecto referencia y el proyecto de la simulación podemos dar contestación a gran parte de los elementos de la rúbrica. Del mismo modo, podemos explotar esta comparación para proporcionar retroalimentación al estudiante de aquellos elementos de la simulación que concuerdan o discrepan con el original.

La evaluación  automática de las competencias en la  gestión de tareas de tareas de un proyecto y del control de versiones es también de  utilidad fuera del ámbito académico. Por ejemplo, podría utilizarse en el proceso de evaluación de competencias técnicas en el mundo empresarial, en las áreas siguientes:

\begin{itemize}
	\item Procesos de selección.
	\item Evaluaciones internas de personal de la compañía.
	\item Formación.
\end{itemize}

En este trabajo, dado que surge de un dominio académico, se abordará la evalución de los casos de estudio desde una perspectiva académica.  Alternativamente, podría también haberse abordado desde la perspectiva de un proceso de control de calidad.