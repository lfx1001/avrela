\capitulo{3}{Conceptos teóricos}

\section{Conceptos de la gestión de proyectos ágiles utilizados en el laboratorio virtual.}

\subsection{Historia de usuario.}

Es una descripción informal de una funcionalidad del sistema a desarrollar. Se documentan desde la perspectiva del usuario final del sistema.

TBD - Incluir captura historia de usuario.

\subsection{Puntos de historia de usuario.}

Es una métrica utilizada por los miembros del equipo de proyecto para expresar el esfuerzo estimado requerido para completar una tarea. En la estimación se considera:

\begin{enumerate}
	\item Complejidad de la tarea.
	\item Cantidad de trabajo de la tarea.
	\item Riesgos, incertidumbres de la tarea a realizar.	
\end{enumerate}

\subsection{Sprint/Milestone/Hito.}

Es un periodo de tiempo durante el cual un trabajo específico, definido en las distintas tareas que componen el sprint. En el laboratorio virtual es de relevancia la siguiente información:

TBD - Incluir captura tarea

\begin{enumerate}
	\item Fecha de inicio.
	\item Fecha de fin.
	\item Puntos de historia de usuario: total de puntos de historia de usuario de las tareas incluidas en el sprint.	
\end{enumerate}

\subsection{Tarea/Issue.}

Cada historia de usuario es dividida en unidades de trabajo individuales. En el laboratorio virtual se plantea especificar la siguiente información:

\begin{enumerate}
	\item Descripción de la tarea.
	\item Comentarios de una tarea: Una tarea puede incluir comentarios, cada comentario tiene un autor.
	\item Miembro del equipo al que se le asigna la tarea.
	\item Puntos de historia de usuario.
	\item Tipos de tarea:
		\begin{enumerate}
			\item Issue: tarea de propósito general.
			\item Bug: incidencia.
		\end{enumerate}
	\item Estados de una tarea:
		\begin{enumerate}
			\item Open/abierta.
			\item Closed/cerrada.
		\end{enumerate}
	\item Etiquetas: permiten la clasificación de una tarea en diversas categorías. Existen etiquetas predefinidas y etiquedas personalizadas definidas por el usuario. A efectos de la simulación se consideran las siguientes etiquetas predefinidas:	
		\begin{enumerate}
			\item Feature/funcionalidad.
			\item Testing/pruebas.
			\item Documentation/documentación.
		\end{enumerate}
\end{enumerate}

TBD - Incluir captura tarea


\section{Secciones}

Las secciones se incluyen con el comando section.

\subsection{Subsecciones}

Además de secciones tenemos subsecciones.

\subsubsection{Subsubsecciones}

Y subsecciones. 


\section{Referencias}

Las referencias se incluyen en el texto usando cite \cite{wiki:latex}. Para citar webs, artículos o libros \cite{koza92}.


\section{Imágenes}

Se pueden incluir imágenes con los comandos standard de \LaTeX, pero esta plantilla dispone de comandos propios como por ejemplo el siguiente:

\imagen{add_remote.png}{Autómata para una expresión vacía}{1}

\section{Listas de items}

Existen tres posibilidades:

\begin{itemize}
	\item primer item.
	\item segundo item.
\end{itemize}

\begin{enumerate}
	\item primer item.
	\item segundo item.
\end{enumerate}

\begin{description}
	\item[Primer item] más información sobre el primer item.
	\item[Segundo item] más información sobre el segundo item.
\end{description}
	
\begin{itemize}
\item 
\end{itemize}

\section{Tablas}

Igualmente se pueden usar los comandos específicos de \LaTeX o bien usar alguno de los comandos de la plantilla.

\tablaSmall{Herramientas y tecnologías utilizadas en cada parte del proyecto corta}{l c}{herramientasportipodeusocorta}
{ \multicolumn{1}{l}{Herramientas} & App AngularJS  \\}{ 
 \makecell{HTML5 More H \\More}& X \\
} 

\tablaSmall{Herramientas y tecnologías utilizadas en cada parte del proyecto}{l c c c c}{herramientasportipodeuso}
{ \multicolumn{1}{l}{Herramientas} & App AngularJS & API REST & BD & Memoria \\}{ 
HTML5 & X & & &\\
CSS3 & X & & &\\
BOOTSTRAP & X & & &\\
JavaScript & X & & &\\
AngularJS & X & & &\\
Bower & X & & &\\
PHP & & X & &\\
Karma + Jasmine & X & & &\\
Slim framework & & X & &\\
Idiorm & & X & &\\
Composer & & X & &\\
JSON & X & X & &\\
PhpStorm & X & X & &\\
MySQL & & & X &\\
PhpMyAdmin & & & X &\\
Git + BitBucket & X & X & X & X\\
Mik\TeX{} & & & & X\\
\TeX{}Maker & & & & X\\
Astah & & & & X\\
Balsamiq Mockups & X & & &\\
VersionOne & X & X & X & X\\
} 
