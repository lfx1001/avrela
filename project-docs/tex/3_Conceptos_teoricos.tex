\capitulo{3}{Conceptos teóricos}

En aquellos proyectos que necesiten para su comprensión y desarrollo de unos conceptos teóricos de una determinada materia o de un determinado dominio de conocimiento, debe existir un apartado que sintetice dichos conceptos.

Algunos conceptos teóricos de \LaTeX \footnote{Créditos a los proyectos de Álvaro López Cantero: Configurador de Presupuestos y Roberto Izquierdo Amo: PLQuiz}.

\section{Secciones}

Las secciones se incluyen con el comando section.

\subsection{Subsecciones}

Además de secciones tenemos subsecciones.

\subsubsection{Subsubsecciones}

Y subsecciones. 


\section{Referencias}

Las referencias se incluyen en el texto usando cite \cite{wiki:latex}. Para citar webs, artículos o libros \cite{koza92}.


\section{Imágenes}

Se pueden incluir imágenes con los comandos standard de \LaTeX, pero esta plantilla dispone de comandos propios como por ejemplo el siguiente:

\imagen{escudoInfor}{Autómata para una expresión vacía}



\section{Listas de items}

Existen tres posibilidades:

\begin{itemize}
	\item primer item.
	\item segundo item.
\end{itemize}

\begin{enumerate}
	\item primer item.
	\item segundo item.
\end{enumerate}

\begin{description}
	\item[Primer item] más información sobre el primer item.
	\item[Segundo item] más información sobre el segundo item.
\end{description}
	
\begin{itemize}
\item 
\end{itemize}

\section{Tablas}

Igualmente se pueden usar los comandos específicos de \LaTeX o bien usar alguno de los comandos de la plantilla.

\tablaSmall{Herramientas y tecnologías utilizadas en cada parte del proyecto}{l c c c c}{herramientasportipodeuso}
{ \multicolumn{1}{l}{Herramientas} & App AngularJS & API REST & BD & Memoria \\}{ 
HTML5 & X & & &\\
CSS3 & X & & &\\
BOOTSTRAP & X & & &\\
JavaScript & X & & &\\
AngularJS & X & & &\\
Bower & X & & &\\
PHP & & X & &\\
Karma + Jasmine & X & & &\\
Slim framework & & X & &\\
Idiorm & & X & &\\
Composer & & X & &\\
JSON & X & X & &\\
PhpStorm & X & X & &\\
MySQL & & & X &\\
PhpMyAdmin & & & X &\\
Git + BitBucket & X & X & X & X\\
Mik\TeX{} & & & & X\\
\TeX{}Maker & & & & X\\
Astah & & & & X\\
Balsamiq Mockups & X & & &\\
VersionOne & X & X & X & X\\
} 
