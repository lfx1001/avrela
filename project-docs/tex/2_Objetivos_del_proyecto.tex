\capitulo{2}{Objetivos del proyecto}

La realización de este proyecto tiene como objetivo principal el desarrollo de un software que permita automatizar la evaluación y retroalimentación de las siguientes actividades planteadas como laboratorio virtual en la asignatura gestión de proyectos del grado en ingeniería informática:

\begin{enumerate}
	\item Gestión de tareas de un proyecto software. 
	\item Simulación de control de versiones de un proyecto open source.
\end{enumerate}

Para acometer lo anterior será necesario dotar al software de las siguientes funcionalidades:

\begin{enumerate}
	\item Obtención de información de la gestión de tareas de un proyecto alojado en GitHub con la extensión ZenHub activada.	
	\item Obtención de información del control de versiones de un proyecto alojado en GitHub.
	\item Comparación de la información de la gestión de tareas de dos proyectos.
	\item Comparación de la información del control de versiones de dos proyectos.			
	\item Aplicación de rúbrica de evalución y de retroalimentación a partir de la información de gestión de tareas y control de versiones de un repositorio.
	\item Aplicación de rúbrica de evalución y de retroalimentación a partir de la comparación de dos repositorios.	 		  
\end{enumerate}

Los objetivos  técnicos del desarrollo serán:

\begin{enumerate}
	\item Uso de técnicas de gestión de proyectos ágiles.
	\item Cumplimiento de estándares de calidad de software y aseguramiento de calidad mediante análisis automatizados con Sonarqube.
	\item Integración con GitHub. 
	\item Integración con ZenHub.
	\item Uso de una arquitectura hexagonal que desacople la funcionalidad desarrollada de los orígenes de datos. Esto facilitará la extensibilidad de la funcionalidad simplificando la  integración con otros gestores de tareas y gestores de repositorios Git.
\end{enumerate}

