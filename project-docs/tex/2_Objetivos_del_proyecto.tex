\capitulo{2}{Objetivos del proyecto}

La realización de este proyecto tiene como objetivo principal el desarrollo de un software que permita automatizar la evaluación y retroalimentación de los siguientes casos de estudio de simulación planteados como laboratorio virtual en la asignatura gestión de proyectos del grado en ingeniería informática:

\begin{enumerate}
	\item Simulación de gestión de tareas de un proyecto software. 
	\item Simulación de control de versiones de un proyecto open source.
\end{enumerate}

Para acometer lo anterior el software desarrollado deberá implementar las siguientes funcionalidades:

\begin{enumerate}
	\item Obtención de información de la gestión de tareas de un proyecto alojado en GitHub con la extensión ZenHub activada:
	\begin{enumerate}
		\item Sprints.
		\begin{enumerate}
			\item Título. 
			\item Fecha de finalización.
			\item Estado.	
		\end{enumerate}	
		\item Tareas.	
		\begin{enumerate}
			\item Descripción. 
			\item Etiquetas.
			\item Puntos de historia.
			\item Comentarios.
			\item Estado.	
		\end{enumerate}	
	\end{enumerate}		
	\item Obtención de información del control de versiones de un proyecto alojado en GitHub:
	\begin{enumerate}
		\item Commits. 
			\begin{enumerate}
				\item Identificador único (SHA, Simple Hashing Algorithm). 
				\item Fecha.
				\item Autor.
			\end{enumerate}		
	\end{enumerate}
	\item Comparación de la información de la gestión de tareas de dos proyectos.
	\item Comparación de la información del control de versiones de dos proyectos.			
	\item Aplicación de rúbrica de evalución  sobre:
	\begin{enumerate}
		\item La información de gestión de tareas y del control de versiones de un repositorio. 
		\item La comparación de la información de gestión de tareas y de control de versiones de dos repositorios, el original y el resultante de la simulación.	
	\end{enumerate}			  
\end{enumerate}

Los objetivos  técnicos del desarrollo son:

\begin{enumerate}
	\item Uso de técnicas de gestión de proyectos ágiles.
	\item Cumplimiento de estándares de calidad de software y aseguramiento de calidad mediante análisis automatizados con Sonarqube.
	\item Integración con GitHub. 
	\item Integración con ZenHub.
	\item Uso de una arquitectura hexagonal \cite{hexagonal-architecture} que desacople la funcionalidad desarrollada de los orígenes de datos. Esto facilitará la extensibilidad de la funcionalidad simplificando la  integración con otros gestores de tareas y gestores de repositorios Git.
\end{enumerate}

